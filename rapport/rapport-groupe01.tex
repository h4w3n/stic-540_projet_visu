\documentclass{article}
\usepackage{graphicx} % Required for inserting images

\title{Projet de visualisation STIC-B540)}
\author{Zaynab DOUDOUH, Bao TRAN, Hà Uyên TRAN }
\date{Novembre 2024}

\begin{document}
	
	\maketitle
	
\section{Introduction}
Dans ce projet, nous avons exploré des ensembles de données portant sur des films, comprenant des informations telles que leur budget, leur popularité, leur genre, et leur pays de production. Les données proviennent de The Movie Database (TMDB) et incluent 23 variables principales, allant des revenus générés à la durée des films, en passant par leur langue originale et leur popularité. Notre objectif principal est d'identifier les facteurs qui influencent la profitabilité, l'appréciation auprès des internautes et la diversité de l'offre des films.

\section{Méthodologie}
Nous avons analysé la profitabilité des films (calculée comme la différence entre les revenus générés et le budget alloué) à travers plusieurs perspectives. Pour chaque question, nous avons utilisé des graphiques créés avec ggplot2 et avons justifié nos choix de représentations visuelles en tenant compte des particularités des données (valeurs extrêmes, distributions non uniformes).

\section*{Q1: L'offre cinématographique}
Dans cette question, nous allons nous intéresser à l’offre cinématographique Cette notion est un concept large qui englobe plusieurs sous-concept comme la diversité des genres, la variations des durées à travers le temps, etc…

\section*{Q2: Facteurs d'appréciation}

\section*{Q3: Facteurs de profitabilité}
	
\section{Annexe}
\subsection*{Q1}

\subsection*{Q2}

\subsection*{Q3}
	
\end{document}